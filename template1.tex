% Options for packages loaded elsewhere
\PassOptionsToPackage{unicode}{hyperref}
\PassOptionsToPackage{hyphens}{url}
\PassOptionsToPackage{dvipsnames,svgnames,x11names}{xcolor}
%
\documentclass[
]{article}

\usepackage{amsmath,amssymb}
\usepackage{lmodern}
\usepackage{iftex}
\ifPDFTeX
  \usepackage[T1]{fontenc}
  \usepackage[utf8]{inputenc}
  \usepackage{textcomp} % provide euro and other symbols
\else % if luatex or xetex
  \usepackage{unicode-math}
  \defaultfontfeatures{Scale=MatchLowercase}
  \defaultfontfeatures[\rmfamily]{Ligatures=TeX,Scale=1}
\fi
% Use upquote if available, for straight quotes in verbatim environments
\IfFileExists{upquote.sty}{\usepackage{upquote}}{}
\IfFileExists{microtype.sty}{% use microtype if available
  \usepackage[]{microtype}
  \UseMicrotypeSet[protrusion]{basicmath} % disable protrusion for tt fonts
}{}
\makeatletter
\@ifundefined{KOMAClassName}{% if non-KOMA class
  \IfFileExists{parskip.sty}{%
    \usepackage{parskip}
  }{% else
    \setlength{\parindent}{0pt}
    \setlength{\parskip}{6pt plus 2pt minus 1pt}}
}{% if KOMA class
  \KOMAoptions{parskip=half}}
\makeatother
\usepackage{xcolor}
\setlength{\emergencystretch}{3em} % prevent overfull lines
\setcounter{secnumdepth}{5}
% Make \paragraph and \subparagraph free-standing
\ifx\paragraph\undefined\else
  \let\oldparagraph\paragraph
  \renewcommand{\paragraph}[1]{\oldparagraph{#1}\mbox{}}
\fi
\ifx\subparagraph\undefined\else
  \let\oldsubparagraph\subparagraph
  \renewcommand{\subparagraph}[1]{\oldsubparagraph{#1}\mbox{}}
\fi

\usepackage{color}
\usepackage{fancyvrb}
\newcommand{\VerbBar}{|}
\newcommand{\VERB}{\Verb[commandchars=\\\{\}]}
\DefineVerbatimEnvironment{Highlighting}{Verbatim}{commandchars=\\\{\}}
% Add ',fontsize=\small' for more characters per line
\usepackage{framed}
\definecolor{shadecolor}{RGB}{241,243,245}
\newenvironment{Shaded}{\begin{snugshade}}{\end{snugshade}}
\newcommand{\AlertTok}[1]{\textcolor[rgb]{0.68,0.00,0.00}{#1}}
\newcommand{\AnnotationTok}[1]{\textcolor[rgb]{0.37,0.37,0.37}{#1}}
\newcommand{\AttributeTok}[1]{\textcolor[rgb]{0.40,0.45,0.13}{#1}}
\newcommand{\BaseNTok}[1]{\textcolor[rgb]{0.68,0.00,0.00}{#1}}
\newcommand{\BuiltInTok}[1]{\textcolor[rgb]{0.00,0.23,0.31}{#1}}
\newcommand{\CharTok}[1]{\textcolor[rgb]{0.13,0.47,0.30}{#1}}
\newcommand{\CommentTok}[1]{\textcolor[rgb]{0.37,0.37,0.37}{#1}}
\newcommand{\CommentVarTok}[1]{\textcolor[rgb]{0.37,0.37,0.37}{\textit{#1}}}
\newcommand{\ConstantTok}[1]{\textcolor[rgb]{0.56,0.35,0.01}{#1}}
\newcommand{\ControlFlowTok}[1]{\textcolor[rgb]{0.00,0.23,0.31}{#1}}
\newcommand{\DataTypeTok}[1]{\textcolor[rgb]{0.68,0.00,0.00}{#1}}
\newcommand{\DecValTok}[1]{\textcolor[rgb]{0.68,0.00,0.00}{#1}}
\newcommand{\DocumentationTok}[1]{\textcolor[rgb]{0.37,0.37,0.37}{\textit{#1}}}
\newcommand{\ErrorTok}[1]{\textcolor[rgb]{0.68,0.00,0.00}{#1}}
\newcommand{\ExtensionTok}[1]{\textcolor[rgb]{0.00,0.23,0.31}{#1}}
\newcommand{\FloatTok}[1]{\textcolor[rgb]{0.68,0.00,0.00}{#1}}
\newcommand{\FunctionTok}[1]{\textcolor[rgb]{0.28,0.35,0.67}{#1}}
\newcommand{\ImportTok}[1]{\textcolor[rgb]{0.00,0.46,0.62}{#1}}
\newcommand{\InformationTok}[1]{\textcolor[rgb]{0.37,0.37,0.37}{#1}}
\newcommand{\KeywordTok}[1]{\textcolor[rgb]{0.00,0.23,0.31}{#1}}
\newcommand{\NormalTok}[1]{\textcolor[rgb]{0.00,0.23,0.31}{#1}}
\newcommand{\OperatorTok}[1]{\textcolor[rgb]{0.37,0.37,0.37}{#1}}
\newcommand{\OtherTok}[1]{\textcolor[rgb]{0.00,0.23,0.31}{#1}}
\newcommand{\PreprocessorTok}[1]{\textcolor[rgb]{0.68,0.00,0.00}{#1}}
\newcommand{\RegionMarkerTok}[1]{\textcolor[rgb]{0.00,0.23,0.31}{#1}}
\newcommand{\SpecialCharTok}[1]{\textcolor[rgb]{0.37,0.37,0.37}{#1}}
\newcommand{\SpecialStringTok}[1]{\textcolor[rgb]{0.13,0.47,0.30}{#1}}
\newcommand{\StringTok}[1]{\textcolor[rgb]{0.13,0.47,0.30}{#1}}
\newcommand{\VariableTok}[1]{\textcolor[rgb]{0.07,0.07,0.07}{#1}}
\newcommand{\VerbatimStringTok}[1]{\textcolor[rgb]{0.13,0.47,0.30}{#1}}
\newcommand{\WarningTok}[1]{\textcolor[rgb]{0.37,0.37,0.37}{\textit{#1}}}

\providecommand{\tightlist}{%
  \setlength{\itemsep}{0pt}\setlength{\parskip}{0pt}}\usepackage{longtable,booktabs,array}
\usepackage{calc} % for calculating minipage widths
% Correct order of tables after \paragraph or \subparagraph
\usepackage{etoolbox}
\makeatletter
\patchcmd\longtable{\par}{\if@noskipsec\mbox{}\fi\par}{}{}
\makeatother
% Allow footnotes in longtable head/foot
\IfFileExists{footnotehyper.sty}{\usepackage{footnotehyper}}{\usepackage{footnote}}
\makesavenoteenv{longtable}
\usepackage{graphicx}
\makeatletter
\def\maxwidth{\ifdim\Gin@nat@width>\linewidth\linewidth\else\Gin@nat@width\fi}
\def\maxheight{\ifdim\Gin@nat@height>\textheight\textheight\else\Gin@nat@height\fi}
\makeatother
% Scale images if necessary, so that they will not overflow the page
% margins by default, and it is still possible to overwrite the defaults
% using explicit options in \includegraphics[width, height, ...]{}
\setkeys{Gin}{width=\maxwidth,height=\maxheight,keepaspectratio}
% Set default figure placement to htbp
\makeatletter
\def\fps@figure{htbp}
\makeatother

\makeatletter
\makeatother
\makeatletter
\makeatother
\makeatletter
\@ifpackageloaded{caption}{}{\usepackage{caption}}
\AtBeginDocument{%
\ifdefined\contentsname
  \renewcommand*\contentsname{Table of contents}
\else
  \newcommand\contentsname{Table of contents}
\fi
\ifdefined\listfigurename
  \renewcommand*\listfigurename{List of Figures}
\else
  \newcommand\listfigurename{List of Figures}
\fi
\ifdefined\listtablename
  \renewcommand*\listtablename{List of Tables}
\else
  \newcommand\listtablename{List of Tables}
\fi
\ifdefined\figurename
  \renewcommand*\figurename{Figure}
\else
  \newcommand\figurename{Figure}
\fi
\ifdefined\tablename
  \renewcommand*\tablename{Table}
\else
  \newcommand\tablename{Table}
\fi
}
\@ifpackageloaded{float}{}{\usepackage{float}}
\floatstyle{ruled}
\@ifundefined{c@chapter}{\newfloat{codelisting}{h}{lop}}{\newfloat{codelisting}{h}{lop}[chapter]}
\floatname{codelisting}{Listing}
\newcommand*\listoflistings{\listof{codelisting}{List of Listings}}
\makeatother
\makeatletter
\@ifpackageloaded{caption}{}{\usepackage{caption}}
\@ifpackageloaded{subcaption}{}{\usepackage{subcaption}}
\makeatother
\makeatletter
\@ifpackageloaded{tcolorbox}{}{\usepackage[many]{tcolorbox}}
\makeatother
\makeatletter
\@ifundefined{shadecolor}{\definecolor{shadecolor}{rgb}{.97, .97, .97}}
\makeatother
\makeatletter
\@ifpackageloaded{sidenotes}{}{\usepackage{sidenotes}}
\@ifpackageloaded{marginnote}{}{\usepackage{marginnote}}
\makeatother
\makeatletter
\makeatother
\ifLuaTeX
  \usepackage{selnolig}  % disable illegal ligatures
\fi
\IfFileExists{bookmark.sty}{\usepackage{bookmark}}{\usepackage{hyperref}}
\IfFileExists{xurl.sty}{\usepackage{xurl}}{} % add URL line breaks if available
\urlstyle{same} % disable monospaced font for URLs
\hypersetup{
  pdftitle={Assignment 1 - Bayesian data analysis},
  pdfauthor={anonymous},
  colorlinks=true,
  linkcolor={blue},
  filecolor={Maroon},
  citecolor={Blue},
  urlcolor={Blue},
  pdfcreator={LaTeX via pandoc}}

\title{Assignment 1 - Bayesian data analysis}
\author{anonymous}
\date{}

\begin{document}
\maketitle
\ifdefined\Shaded\renewenvironment{Shaded}{\begin{tcolorbox}[interior hidden, frame hidden, borderline west={3pt}{0pt}{shadecolor}, enhanced, sharp corners, boxrule=0pt, breakable]}{\end{tcolorbox}}\fi

\hypertarget{general-information}{%
\section{General information}\label{general-information}}

\hypertarget{basic-probability-theory-notation-and-terms}{%
\section{Basic probability theory notation and
terms}\label{basic-probability-theory-notation-and-terms}}

\textbf{Write your answers here!}

\hypertarget{basic-computer-skills}{%
\section{Basic computer skills}\label{basic-computer-skills}}

\textbf{Do some setup here. Explain shortly what you do.}

\begin{Shaded}
\begin{Highlighting}[]
\CommentTok{\# Do some setup:}
\NormalTok{distribution\_mean }\OtherTok{=}\NormalTok{ .}\DecValTok{2}
\NormalTok{distribution\_variance }\OtherTok{=}\NormalTok{ .}\DecValTok{01}
\CommentTok{\# You have to compute the parameters below from the given mean and variance}
\CommentTok{\# distribution\_alpha = ...}
\CommentTok{\# distribution\_beta = ...}
\end{Highlighting}
\end{Shaded}

\hypertarget{a}{%
\subsection{(a)}\label{a}}

\textbf{Plot the PDF here. Explain shortly what you do.}

\begin{Shaded}
\begin{Highlighting}[]
\CommentTok{\# Useful functions: seq(), plot() and dbeta()}
\end{Highlighting}
\end{Shaded}

\hypertarget{b}{%
\subsection{(b)}\label{b}}

\textbf{Draw samples and plot the histogram here. Explain shortly what
you do.}

\begin{Shaded}
\begin{Highlighting}[]
\CommentTok{\# Useful functions: rbeta() and hist()}
\end{Highlighting}
\end{Shaded}

\hypertarget{c}{%
\subsection{(c)}\label{c}}

\textbf{Compute the sample mean and variance here. Explain shortly what
you do.}

\begin{Shaded}
\begin{Highlighting}[]
\CommentTok{\# Useful functions: mean() and var()}
\end{Highlighting}
\end{Shaded}

\hypertarget{d}{%
\subsection{(d)}\label{d}}

\textbf{Compute the central interval here. Explain shortly what you do.}

\begin{Shaded}
\begin{Highlighting}[]
\CommentTok{\# Useful functions: quantile()}
\end{Highlighting}
\end{Shaded}

\hypertarget{bayes-theorem-1}{%
\section{Bayes' theorem 1}\label{bayes-theorem-1}}

\textbf{Compute the quantities needed to justify your recommendation
here. Explain shortly what you do.}

\begin{Shaded}
\begin{Highlighting}[]
\CommentTok{\# You can do the computation with pen and paper or in R. }
\CommentTok{\# Either way, you have to explain why you compute what you compute.}
\end{Highlighting}
\end{Shaded}

\hypertarget{bayes-theorem-2}{%
\section{Bayes' theorem 2}\label{bayes-theorem-2}}

\textbf{You will need to change the numbers to the numbers in the
exercise.}

\begin{Shaded}
\begin{Highlighting}[]
\NormalTok{boxes\_test }\OtherTok{\textless{}{-}} \FunctionTok{matrix}\NormalTok{(}\FunctionTok{c}\NormalTok{(}\DecValTok{2}\NormalTok{,}\DecValTok{2}\NormalTok{,}\DecValTok{1}\NormalTok{,}\DecValTok{5}\NormalTok{,}\DecValTok{5}\NormalTok{,}\DecValTok{1}\NormalTok{), }\AttributeTok{ncol =} \DecValTok{2}\NormalTok{,}
    \AttributeTok{dimnames =} \FunctionTok{list}\NormalTok{(}\FunctionTok{c}\NormalTok{(}\StringTok{"A"}\NormalTok{, }\StringTok{"B"}\NormalTok{, }\StringTok{"C"}\NormalTok{), }\FunctionTok{c}\NormalTok{(}\StringTok{"red"}\NormalTok{, }\StringTok{"white"}\NormalTok{)))}
\end{Highlighting}
\end{Shaded}

\hypertarget{a-1}{%
\subsection{(a)}\label{a-1}}

\textbf{Keep the below name and format for the function to work with
\texttt{markmyassignment}:}

\begin{Shaded}
\begin{Highlighting}[]
\NormalTok{p\_red }\OtherTok{\textless{}{-}} \ControlFlowTok{function}\NormalTok{(boxes) \{}
    \CommentTok{\# Do computation here, and return as below.}
    \CommentTok{\# This is the correct return value for the test data provided above.}
    \FloatTok{0.3928571}
\NormalTok{\}}
\end{Highlighting}
\end{Shaded}

\hypertarget{b-1}{%
\subsection{(b)}\label{b-1}}

\textbf{Keep the below name and format for the function to work with
\texttt{markmyassignment}:}

\begin{Shaded}
\begin{Highlighting}[]
\NormalTok{p\_box }\OtherTok{\textless{}{-}} \ControlFlowTok{function}\NormalTok{(boxes) \{}
    \CommentTok{\# Do computation here, and return as below.}
    \CommentTok{\# This is the correct return value for the test data provided above.}
    \FunctionTok{c}\NormalTok{(}\FloatTok{0.29090909}\NormalTok{,}\FloatTok{0.07272727}\NormalTok{,}\FloatTok{0.63636364}\NormalTok{)}
\NormalTok{\}}
\end{Highlighting}
\end{Shaded}

\hypertarget{bayes-theorem-3}{%
\section{Bayes' theorem 3}\label{bayes-theorem-3}}

\textbf{You will need to change the numbers to the numbers in the
exercise.}

\begin{Shaded}
\begin{Highlighting}[]
\NormalTok{fraternal\_prob }\OtherTok{=} \DecValTok{1}\SpecialCharTok{/}\DecValTok{125}
\NormalTok{identical\_prob }\OtherTok{=} \DecValTok{1}\SpecialCharTok{/}\DecValTok{300}
\end{Highlighting}
\end{Shaded}

\textbf{Keep the below name and format for the function to work with
\texttt{markmyassignment}:}

\begin{Shaded}
\begin{Highlighting}[]
\NormalTok{p\_identical\_twin }\OtherTok{\textless{}{-}} \ControlFlowTok{function}\NormalTok{(fraternal\_prob, identical\_prob) \{}
    \CommentTok{\# Do computation here, and return as below.}
    \CommentTok{\# This is the correct return value for the test data provided above.}
    \FloatTok{0.4545455}
\NormalTok{\}}
\end{Highlighting}
\end{Shaded}




\end{document}
